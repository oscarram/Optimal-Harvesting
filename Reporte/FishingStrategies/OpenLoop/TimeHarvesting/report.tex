Given a time horizon $T$, the amount of fish extracted for the end of this time, is given by
\begin{equation}
	J(u;T)=\int_{0}^{T} u(t)\diff{t}
\end{equation}
Where $u(t)$ is the harvesting rate at time $t$, introduced to the population as follows,
\begin{equation}
	\dev{x}{t}=rx\left(1-\frac{x}{M}\right)-u(t) \label{eq: HarvestOpenLoop}
\end{equation}
We would like to know among all the possible time functions $u\in \mathbb{R}^{[0,T]}$, the functional $J$ is maximized. 

Stating in this way, is not well posed. Since we see that the functional is linear in $u$, implying the bigger gets $u$, the bigger gets $J$. But this approach has the disadvantage we showed in the previous discussion, that the bigger $u$ the fastest our population can lead to extinction and therefore, we will not be able to continue harvesting the pool. 

From the previous results we see that for $u(t)>\frac{rM}{4}$, we lead the population to extinction. Therefore, we restrict our harvesting rate $u$ to,
\begin{equation}
0\leq u(t)\leq\frac{rM}{4}.
\end{equation}

Please note that,
