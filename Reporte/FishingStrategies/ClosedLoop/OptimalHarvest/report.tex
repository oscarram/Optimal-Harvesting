Since our harvesting control is proportional to our population, given a finite time horizon $T$, the amount of fishes we have extracted from our pool is given by,
\begin{align}
	J(x;p,T)&=\int_{0}^{T} p x \diff{t} \nonumber \\
			&=\int_{0}^{T} \frac{M p\left(r-p\right)x_0}{rx_0+\left(M\left(r-p\right)-rx_0\right)\euler^{-(r-p) t}} \diff{t} \nonumber
\end{align}
The equation \ref{eq: Proportional Population} determines the population of fishes at time $t$. Consider the transformations $y=x/M$, $\tau=rt$, $\overline{p}=rp$. Therefore the equation \ref{eq: PropHarvest}, is transformed into:
\begin{equation}
\dev{y}{\tau}=(1-\overline{p})y\left(1-\frac{y}{1-\overline{p}}\right)
\end{equation}
with initial condition $y(0)=y_0=x_0/M$. And solution,

\begin{equation}
	y(\tau)=\frac{(1-\overline p)y_0}{y_0+\left(1-\overline p-y_0\right)\euler^{(\overline p-1)\tau}}
\end{equation}
Then our function in the time horizon $\overline{T}=rT$
\begin{align}
J\left(y;\overline{p},\overline{T}\right)&=\frac{1}{rM}\int_0^{\overline{T}}\overline{p}y(\tau)\diff{\tau}	\\
&=\frac{\overline{p}}{rM}\left(\ln\left(1-\overline{p}+y_0\left(\euler^{(1-\overline{p})\overline{T}}-1\right)\right)-\ln\left(1-\overline{p}\right)\right)
\end{align}

We would like to know the constant $\overline{p}^*$ that for a given time horizon $\overline{T}$ maximizes $J$. Therefore $\overline{p}$ should satisfy the necessary condition,
\begin{equation}
\left.\pdev{J\left(y;\overline{p},\overline{T}\right)}{p}\right|_{\overline{p}=\overline{p}^*}=0
\end{equation}

Therefore, for given $y_0$ we need to solve for $\overline{p}^*$ the following equation,
\begin{equation}
\overline{p}^* \left(\frac{1+Ty_0\euler^{\left(1-\overline{p}^*\right)\overline{T}}}{\overline{p}^*+y_0-1 -y_0\euler^{(1-\overline{p}^*)\overline{T}}}+\frac{1}{1-\overline{p}^*}\right)+\ln
\left(1-\overline{p}^*-y_0 +y_0\euler^{(1-\overline{p}^*)\overline{T}}\right)-\ln\left(1-\overline{p}^*\right)=0 \label{eq: Nonlinear equation}
\end{equation}

This expression has no closed form solution, but we can estimate it numerically, if we know $y_0$ and $T$. For example for $y_0=0.75$ and $\overline{T}=20$, we have
$\overline{p}^* \approx 0.541881$.


We can know the value of $\overline{p}^*$, for $T\rightarrow\infty$, Consider,  
\begin{equation}	
\lim\limits_{T\rightarrow \infty} \ln(a+b\euler^{cT})\approxeq cT
\end{equation}

For any $a,b \in \mathbb{R}$. And
\begin{equation}
	\lim\limits_{T-\infty}\frac{a+bT\euler^{cT}}{r+d\euler^{cT}} \approxeq \frac{b}{d}T
\end{equation}
For any constants $a,b,c,d,r \in \mathbb{R}$. 

Therefore, for big enough $T$, the contribution for fixed $\overline{p}^*$, we can write equation \ref{eq: Nonlinear equation} in small $o$ notation as follows,
\begin{equation}
\lim\limits_{T\rightarrow \infty}\left.\pdev{J}{p}\right|_{p=\overline{p}^*} =(1-\overline{p}^*)\overline{T}-\overline{p}^*\overline{T}+o(T)+o(T^2)+\dots=0 
\end{equation}

Hence, when $T\rightarrow \infty$
\begin{equation}
	(1-2\overline{p}^*)\overline{T}=0\implies \overline{p}^*=\frac{1}{2}
\end{equation}
